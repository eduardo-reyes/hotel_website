\section{\bfseries\LARGE Casos de Uso: Cliente}

\begin{enumerate}
    \item Registrarse. Un nuevo usuario puede registrar sus datos para crear una nueva cuenta, necesita introducir su nombre, apellidos, correo electrónico y una contraseña.
    \item Iniciar Sesión. Un usuario ya registrado en el sistema puede iniciar sesión en su cuenta usando su correo electrónico y su contraseña.
    \item Buscar Hoteles. El usuario puede buscar hoteles para reservar o consultar precios. Necesita ingresar el nombre del hotel o la ubicación, las fechas de entrada y salida, el número de habitaciones y el número de huéspedes por habitación.
    \item Ver sus reservaciones. El usuario que haya ingresado a su cuenta podrá ver un listado de sus reservaciones futuras y otro de sus reservaciones pasadas.
    \item Crear reservación. El usuario podrá crear una nueva reservación en alguno de nuestros hoteles para las fechas especificadas y para el número de huéspedes. 
    \item Cancelar reservación. Si el usuario lo requiere puede cancelar su reservación.
    \item Valorar hotel. El usuario puede darle una calificación al hotel con estrellitas del 1-5, donde 1 es una experiencia mala y 5 es una experiencia perfecta.
    \item Publicar reseña. El usuaio puede añadir comentarios sobre su estancia, sobre las instalaciones del hotel o los servicios recibidos durante ella para así ayudar a otros huéspedes a tener una idea sobre lo qué se debe esperar del hotel. De igual manera para ayudar al hotel a mejorar en sus áreas de oportunidad.
    \item Eliminar reseña. El usuario puede, si así lo desea, eliminar su reseña.
    \item Cerrar sesión. El usuario puede cerrar su sesión si ya no necesita realizar ninguna acción por el momento. El sistema también puede cerrar sesión por inactividad.
\end{enumerate}


\section{\bfseries\LARGE Casos de Uso: Trabajador}
\begin{enumerate}
    \item Iniciar sesión. Un trabajador puede iniciar sesión en su cuenta usando su correo electronico institucional y su contraseña.
    \item Registrar usuarios. Un trabajador que tenga los suficientes permisos puede dar de alta a otros trabajadores en el sistema. Necesitará los siguientes datos del trabajador: nombre completo, teléfono de contacto, correo electrónico institucional y una contraseña provisional.
    \item Buscar hoteles. Un trabajador puede buscar hoteles para reservar o consultar precios. Necesita ingresar el nombre del hotel o la ubicación. En caso de querer reservar o consultar precios, necesita ingresar las fechas de entrada y salida, el número de habitaciones y el número de huéspedes por habitación. Si el trabajador tiene los permisos necesarios, podrá consultar información pertienente sobre el funcionamiento del hotel: reservaciones activas, reservaciones canceladas, reservaciones realizadas, ver el estado en tiempo real de los trabajadores de un hotel en específico, etc.
    \item Reservaciones. Un trabajador, al igual que un usuario, podrá crear y cancelar reservaciones, así como ver sus reservaciones pasadas.
    \item Atender solicitudes. Un trabajador puede atender solicitudes de clientes siempre y cuando pertenezcan a sus áreas de trabajo. Por ejemplo, solicitud de Room Service.
    \item Definir estado de trabajo. Un trabajador puede definir su estado actual de trabajo: activo, inactivo, ocupado, etc.
    \item Responder reseñas. Un trabajador que tenga los suficientes permisos podrá responder a las reseñas y/o comentarios de los clientes sobre su estadía en algún hotel en específico.
    \item Habilitar o deshabilitar hotel y sus funciones. Un usuario que tenga los suficientes permisos puede habilitar o deshabilitar hoteles (por ejemplo, por cuestiones de remodelaje o reparaciones), así como deshabilitar funciones o secciones específicas dentro de un hotel (por ejemplo, deshabilitar el servicio de restaurantes por alguna causa de fuerza mayor).
    \item Cerrar sesión. El trabajador puede cerrar sesión si ya no necesita realizar ninguna acción por el momento. El sistema no puede cerrar sesión por inactividad.
\end{enumerate}